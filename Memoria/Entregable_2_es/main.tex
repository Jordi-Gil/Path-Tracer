\documentclass[titlepage,12pt]{report}

\usepackage[utf8]{inputenc}
\usepackage[T1]{fontenc}
\usepackage{microtype}
\usepackage[dvips]{graphicx}
\usepackage{xcolor}
\usepackage{times}


\usepackage[
    pdftitle={Títol},
    pdfkeywords={Ray tracing, Path tracing, Computer Graphics},
    pdfauthor={Jordi Gil},
    breaklinks=true,colorlinks=true,
    linkcolor=black,urlcolor=black,citecolor=black,
    bookmarks=true,bookmarksopenlevel=2
]{hyperref}

%~ \usepackage{geometry}
% PDF VIEW
%~ \geometry{total={210mm,297mm},
%~ left=25mm,right=25mm,
%~ bindingoffset=0mm, top=25mm,bottom=25mm}
% PRINT
%~ \geometry{total={210mm,297mm},
%~ left=20mm,right=20mm,
%~ bindingoffset=10mm, top=25mm,bottom=25mm}

\OnehalfSpacing
\linespread{1.2}

%%% CHAPTER'S STYLE
%~ \chapterstyle{bianchi}
%~ \chapterstyle{ger}
%~ \chapterstyle{madsen}
%~ \chapterstyle{tandh}
%~ \chapterstyle{ell}

%~ \chapterstyle{komalike}
\chapterstyle{tandh}
%~ \chapterstyle{default}



%%% STYLE OF SECTIONS, SUBSECTIONS, AND SUBSUBSECTIONS


%%% STYLE OF PAGES NUMBERING
\makepagestyle{plain}
\makeevenfoot{plain}{}{\thepage}{}
\makeoddfoot{plain}{}{\thepage}{}
\makeevenhead{plain}{}{}{}
\makeoddhead{plain}{}{}{}

\maxsecnumdepth{subsection} % chapters, sections, and subsections are numbered
\maxtocdepth{subsection} % chapters, sections, and subsections are in the Table of Contents



\setsecheadstyle{\Large   \normalfont\sffamily\bfseries}
\setsubsecheadstyle{\normalfont\sffamily\bfseries}


\begin{document}

\iftrue
\newcommand{\HRule}{\rule{\linewidth}{0.5mm}}

\thispagestyle{empty}

\begin{center}

{\large Universitat Politècnica de Catalunya}

\medskip
{\large Facultat d'Informàtica de Barcelona (FIB)}

\vfill
{\bfseries\Large Trabajo de Fin de Grado}

\vfill
\centerline{\mbox{\includegraphics[width=60mm]{media/FIB_UPC.png}}}

\vfill
\vspace{5mm}

{\LARGE Jordi Gil González}

\vspace{15mm}

% Title in English according to the official assignment
{\LARGE\bfseries Análisis del método de renderizado \\ Path Tracing en CPU y GPU}

\normalfont \small \sffamily{Informe de seguimiento}

\vfill

Departament de Ciències de la Computació


\vfill

\begin{tabular}{rl}
Director del Trabajo de Fin de Grado: & Chica Calaf, Antoni \\
\noalign{\vspace{2mm}}
Programa de estudio: & Computación\\
\noalign{\vspace{2mm}}
Especialización: & Computación Gráfica\\
\end{tabular}

\vfill

\large Curso Académico 2019/2020

\large \today

\end{center}

\newpage
\tableofcontents*
\fi

\newpage

\chapter{Planificación temporal}

\section{Planificación y programación}

Este capítulo trataremos la planificación con el objetivo de describir las tareas que serán llevadas a cabo con el fin de realizar el presente proyecto, brindando de esta forma una plan de acción que defina las diferentes acciones para el cumplimiento del proyecto en el tiempo estimado. Veremos también los recursos utilizados y se tendrán también en cuenta posibles obstáculos que puedan modificar la planificación.

El proyecto inició a principios de Julio de 2019 con fecha límite el 13 de Enero de 2020.

\section{Descripción de tareas y recursos utilizados}

\subsection{Descripción de tareas}

\subsubsection{Estudio de conceptos}

Antes de comenzar el proyecto, fue necesario familiarizarse con los conceptos básicos que rigen nuestro proyecto. Durante el semestre previo hubo un para de reuniones con el director del proyecto para definir el tema en el cual se centraría y poder así encaminarlo. También, el autor del presente proyecto se matriculó de la asignatura de Tarjetas Gráficas y Aceleradores (TGA por sus siglas) para introducirse en el entorno de \texttt{CUDA} y así agilizar el proceso de desarrollo del proyecto en el momento de su inicio. Dado que esta tarea se realiza fuera del proyecto y es difícil de determinar una estimación en horas, no se tendrá en cuenta.

\subsubsection{Configuración del sistema}

Antes de entrar en el desarrollo en si del proyecto, debemos configurar las herramientas necesarias y probar que funcionan correctamente.

Para poder desarrollar el proyecto será necesario tener instalado en nuestros sistemas la API de \texttt{CUDA} con tal de poder crear el programa principal de nuestra aplicación. Las librerías de \texttt{OpenMP} vienen instaladas por defecto en los sistemas Linux por lo que no será necesaria una instalación, pero si comprobar que funcionan correctamente. Por último paras uso de una plataforma, \textit{TexMaker}, que nos permita compilar el código \LaTeX con tal de generar la documentación.

Esta configuración debe hacerse tanto en el computador de sobremesa como en el portátil. La parte de \textit{CUDA} en el cluster de docencia no ser'a necesario de configurar debido a que ya est'a previamente configurado por el DAC (Departament d'Arquitectura de Computadors). 

\subsubsection{Planificación del proyecto}

Esta es la tarea que se está desarrollando actualmente. Esencialmente trata sobre todo el contenido cubierto por el curso de GEP. La podemos dividir en tres sub-tareas:

\begin{enumerate}
		\item Contextualización y alcance.
		\item Planificación temporal.
		\item Presupuesto y sostenibilidad.
\end{enumerate}

\subsubsection{Desarrollo}

Esta tarea es la más importante de todo el proyecto. Cubre tanto el desarrollo de la aplicación, experimentación y documentación de los resultados de cada una de las partes.
Podemos dividir el desarrollo en tres etapas: 

\begin{itemize}

	\item \textbf{Desarrollo versión secuencial:} Se trata de la versión más básica de todas. Sirve como base para desarrollar tanto la versión paralela en CPU (\texttt{OpenMP}), como la versión paralela en GPU (\texttt{CUDA}).
	
	\item \textbf{Desarrollo versión paralela CPU:} Una vez implementada la versión secuencial pasaremos a implementar una versión paralela de esta haciendo uso de la librería de \texttt{OpenMP}. 
	
	\item \textbf{Dearrollo versión paralela GPU:} Una vez implementada la versión secuencia pasaremos a implementar una versión paralela en \texttt{CUDA}.

	\item \textbf{Experimentación:} Crearemos la misma escena en cada una de las versiones comentadas en los puntos anteriores y analizaremos el rendimiento de ambas, teniendo en cuenta el tiempo de creación requerido para generar la imagen final.

\end{itemize}

Estas etapas no son secuenciales. Es decir, a medida que vayamos desarrollando la aplicación secuencial, antes de añadir características importantes, se implementará tanto la versión de CPU como la versión de GPU. Es por eso que cada una de las etapas se realizarán más de una vez en el curso del desarrollo del proyecto.

Cada vez que implementemos una nueva versión de cada una de las aplicaciones, éstas serán testeadas y comparadas entre si. De esta forma podremos ir haciendo los experimentos pertinentes y ver así como es el rendimiento de nuestras aplicaciones con las distintas características que iremos añadiendo. Un ejemplo de ello es comparar como afecta al rendimiento el uso de estructura de datos aceleradoras (en todas las versiones), frente a una versión en la cual no se hace uso de este tipo de estructuras de datos y así justificar el uso de ellas.

Las dependencias entre ellas son fáciles de describir. Hasta no tener la primera versión secuencial no podremos empezar a programar las versiones paralelas. Tampoco podremos añadir mejoras a la versión secuencial sin tener implementadas las versiones paralelas pertinentes.

\subsubsection{Etapa final}

En esta etapa realizaremos toda la documentación de las etapas anteriores y preparación de la presentación de la defensa del presente proyecto.

\subsection{Tabla resumen}

\begin{table}
	\centering
	\begin{tabular}{|m{5cm}||m{5cm}|}
		\hline
		Tarea & Tiempo empleado (horas) \\ \hline \hline
		Configuración del sistema & 10 \\ \hline
		Planificación del proyecto & 90 \\ \hline
		Desarrollo & 380 \\ \hline
		Etapa final & 50 \\ \hline \hline
		Total & 530 \\ \hline
	\end{tabular}
\end{table}

\begin{table}
	\centering
	\begin{tabular}{|m{5cm}||m{5cm}|}
		\hline
		Tarea & Tiempo empleado (horas) \\ \hline \hline
		Desarrollo secuencial & 80 \\ \hline
		Desarrollo CPU & 90 \\ \hline
		Desarrollo GPU & 105 \\ \hline
		Experimentos & 105 \\ \hline \hline
		Total & 380 \\ \hline
	\end{tabular}
\end{table}

\subsection{Recursos utilizados}

Podemos dividir los recursos necesitados para la realización de las tareas en: hardware y software. A continuación tenemos ambas listas y las tareas en las cuales son utilizadas.

\subsubsection{Software}

\begin{itemize}
	\item Linux, usado en todas las tareas.
	\item \LaTeX , usado para realizar la documentación.
	\item \texttt{C++},\texttt{OpenMP} y \texttt{CUDA}, usado en el desarrollo de la aplicación.
	\item \texttt{git}, usado para el control de versiones y compartir de código entre diferentes entornos.
\end{itemize}

\subsubsection{Hardware}

\begin{itemize}
	\item PC Sobremesa con las siguientes características:
		\begin{itemize}
			\item i7 7700 3.6 GHz
			\item NVIDIA GeForce RTX 2080 SUPER
			\item 24GB RAM
			\item 250 SSD, 1TB SSD, 1TB HDD
		\end{itemize}
	\item PC Portátil con las siguientes características:
		\begin{itemize}
			\item i7 7700HQ 2.8GHz
			\item NVIDIA GeForce 1050 Mobile
			\item 8GB RAM
			\item 500GB SSD
		\end{itemize}
	\item Cluster Docencia
		\begin{itemize}
			\item Intel Xeon E5-2620 v2 2.10GHz x2
			\item NVIDIA Tesla K40c x4
			\item 64GB RAM 
			\item 1TB x2
		\end{itemize}
\end{itemize}


\section{Diagrama de Gannt y Estimación}

\section{Gestión de riesgos: Planes alternativos y obstáculos}


\end{document}
